% das Papierformat zuerst
\documentclass[a4paper, 11pt]{article}

\usepackage[utf8]{inputenc} % Kodierung
\usepackage[ngerman]{babel} % Sprache
\usepackage{graphicx}  % Bildchen
\usepackage{float}  % Bildchen2
\usepackage{rotating} %Bildchen3

% wir wollen auf jeder Seite eine Ueberschrift
\pagestyle{headings}

% hier beginnt das Dokument
\begin{document}

\thispagestyle{empty}
\begin{center}
\Large{Karlsruher Institut für Technologie}\\
\end{center}

\begin{center}
\Large{Fakultät für Wirtschaftswissenschaften}
\end{center}
\begin{verbatim}





\end{verbatim}
\begin{center}
\textbf{\LARGE{Seminararbeit}}
\end{center}
\begin{verbatim}


\end{verbatim}
\begin{center}
\textbf{am Institut für Angewandte Informatik und Formale Beschreibungsverfahren}
\end{center}
\begin{verbatim}


\end{verbatim}

\begin{verbatim}









\end{verbatim}
\begin{flushleft}
\begin{tabular}{lll}
\textbf{Thema:} & & Linked Open Data basierte Web 3.0 Anwendungen \\
& & \\
& & \\
\textbf{eingereicht von:} & & Xinjie Du\\
& & Christoph Gielisch \\
& & Andreas Gutzan \\
& & Clemens Stolle \\
& & \\
\textbf{eingereicht am:} & & 18. Juli 2011\\
& & \\
\textbf{Betreuer:} & & Herr Prof. Dr. Rudi Studer \\
& & Herr Dipl.-Wirt.-Ing. Daniel Herzig \\
& & Herr Dipl.-Inf. Benedikt Kämpgen \\
& & Herr Dipl.-Inf. Günter Ladwig
\end{tabular}
\end{flushleft}

\begin{abstract}Das Internet umfasst eine riesige Menge an Informationen jeglicher Art. Da diese aber meist in einer unstrukturierten Weise vorliegen, ist es schwierig Daten aus verschiedenen Quellen miteinander zu verknüpfen. Hier soll Linked Open Data Abhilfe schaffen. Durch diverse Beschreibungs- und Abfragesprachen können Informationen strukturiert und standardisiert gespeichert und abgefragt werden. Dadurch wird die maschinelle Informationsverarbeitung erheblich vereinfacht.\\\\
Eurotrip ist ein Allgemeinwissen- und Geographiequiz, das mehrere Linked Open Data Datensätze verwendet, um immer wieder unterschiedliche Fragen zu generieren. Da dem Spiel kein fester Fragenkatalog zu Grunde liegt, existiert theoretisch eine unbegrenzte Anzahl an Fragemöglichkeiten. Es werden über spezielle Abfragen mehrere Quellen so miteinander verknüpft, dass eine Frage-Antwort Kombination mit Bildern, Texten und geographischen Daten entsteht, die es in dieser Form noch nicht gibt.\\\\
Mit jeder Frage generiert der Spieler einen neuen Datensatz für einen Ort, der Informationen wie lokale Sehenswürdigkeiten, dazugehörige Fotos, die Landesflagge und Verweise auf andere Linked Open Data Ressourcen enthält. Dieser Datensatz wird in strukturierter Form gespeichert, so dass eine spätere Weiterverwendung der neu verknüpften Daten durchaus denkbar ist. \\\\
In spielerischer Form wird so die Linked Open Data Cloud mit neuen Querverweisen und Informationsverknüpfungen angereichert.
\end{abstract}
\thispagestyle{empty}
\newpage
\tableofcontents
\setcounter{page}{1}
\pagenumbering{Roman}

\newpage
\setcounter{page}{1}
\pagenumbering{arabic}
\section{Einleitung}
\subsection{Problemdefinition}
Neben einer Vielzahl von unstrukturierten Daten entsteht im World Wide Web eine große Cloud mit frei verfügbaren Daten,  die per URI kodiert und verlinkt sind. Diese Linked Open Data sind Teil des Semantic Web und besitzt gegenüber konventioneller Datenrepräsentation viele Vorteile.\\\\
Das folgende Projekt \textit{Eurotrip} entsteht in Gruppenarbeit als Teil des Seminarpraktikums \textit{Linked Open Data basierte Web 3.0 Anwendungen}. \\\\
Es hat das Ziel einen lauffähigen Prototyp einer LOD Anwendung zu erstellen, der mindestens zwei LOD-Datensätze verwendet und dessen Verwendung einen direkten Nutzen aus diesen Datensätzen zieht. Das Ergebnis soll so flexibel gehalten werden, dass weitere Datensätze potentiell integriert werden können. 
\subsection{Herangehensweise und Ziele}
In den folgenden Kapiteln wird sowohl die Beschreibung  der Projektidee als auch die technische Umsetzung geschildert. Ebenso wird kurz auf die Projektplanung eingegangen. Der Fokus liegt speziell auf der Evaluierung der Vorteilhaftigkeit bei der Verwendung von LOD. \\\\
Wichtig war uns, dass die Projektidee zwar innovativ aber auch gleichzeitig im technischen sowie zeitlichen Rahmen realisierbar sein muss. Das Ziel sollte dabei primär auf der Verwendung von mehreren LOD-Datensätzen liegen, deren Benutzung die Vorteile des Semantischen Webs ermöglicht. Des Weiteren wird Wert auf die flexible Erweiterungsmöglichkeit gelegt. \\\\
Dem Projektteam ist bewusst, dass das Ergebnis kein marktfähiges Produkt sondern lediglich ein Prototyp darstellen kann. Einbußen bei der Bedienerfreundlichkeit, Geschwindigkeit sowie Fehlerfreiheit werden notwendigerweise in Kauf genommen. Schwächen und Schwierigkeiten bei der Verwendung von LOD werden explizit den Stärken gegenübergestellt.\\\\
Dieser Arbeit liegt eine CD bei, die den Quellcode des kompletten Projektes beinhaltet. Der komplette Sourcecode kann aber auch online angesehen werden.\footnote{https://github.com/m0l0t0ph/Eurotrip}
\newpage
\section{Beschreibung der Idee}
\subsection{Anwendungszenario}
Für die Anwendbarkeit des Prototyps gilt es zunächst zwei verschiedene Grundüberlegungen zu separieren. Es stellt sich die Frage, welchen Nutzen das Produkt zum einen für den potentiellen Anwender, also den Spieler, und zum anderen für den Entwickler bzw. den Vertreibenden bietet.\\\\
Für den Anwender ist Eurotrip ein Quiz- oder Lernspiel. Abgefragt werden hauptsächlich geografische Kenntnisse. Dabei sorgt eine Punktevergabe für einen kompetitiven Faktor. Das Spiel positionier sich somit sowohl als Edutainment-Software\footnote{http://www.hdm-stuttgart.de/ifak/medientipps/edutainment/definition/} als auch als Unterhaltungssoftware für die kurzweilige Ablenkung, z.B. als Facebook-Spiel.\\\\
Als Anbieter der Software ist neben der Schaffung einer Einnahmemöglichkeit über Werbeeinblendungen oder Verkauf der Software vor Allem die Generierung von strukturierten Daten interessant. 
\subsection{Spielablauf}
Der Spielablauf gliedert sich in zehn Fragerunden. In jeder Fragerunde sucht das Programm drei Fotografien zu einer europäischen Stadt, die gewissen Ansprüchen genügen muss\footnote{siehe getCity.php in Kapitel \ref{sec:getCity}}, heraus. Der Spieler muss nun versuchen diese Stadt mit Hilfe der Fotografien zu erkennen und sie auf einer Europakarte mit einem Mausklick möglichst genau lokalisieren. Er bekommt dabei weniger Punkte je weiter sein Tipp vom richtigen Zielort entfernt liegt, wobei ab einer gewissen Entfernung pauschal null Punkte vergeben werden.\\\\ Weiterhin kann er sich nach Bedarf in jeder Fragerunde im Tausch gegen Punkte drei neue Fotografien oder auch einen 200 Zeichen langen Kurzhinweis anzeigen lassen. Wenn der Spieler mit seinem Tipp eine gewisse Maximalentfernung nicht überschreitet, bekommt er darüber hinaus eine Bonusfrage zur aktuellen Stadt oder deren Land gestellt. Diese vermittelt zusätzliche Hintergrundinformationen und gibt dem Spieler die Möglichkeit ein paar Bonuspunkte zu erspielen.\\\\ 
Die so in jeder Fragerunde erspielten Punkte werden kumuliert. Ziel des Spiels ist es nach zehn Fragerunden einen möglichst hohen Punktestand zu erreichen.
\subsection{Alleinstellungsmerkmal und Abgrenzung}
Die grundlegende Spielidee wurde schon in Facebook- sowie iPhone-Apps implementiert\footnote{http://www.facebook.com/apps/application.php?id=134046623315935, \\http://itunes.apple.com/de/app/georific/id320207678?mt=8}. Allerdings setzen diese zur Erzeugung der Fragen auf statische Datenbanken und sind somit im Fragenumfang limitiert. Durch den Einsatz von LOD konnte hier eine dynamische, sich selbstständig aktualisierende Variante geschaffen werden. Des Weiteren sorgt der Aufbau des Prototyps, gerade in Bezug auf die Bonusfragen, für eine modularisierte Erweiterbarkeit des bestehenden Programmes mit anderen Datenquellen im Semantic Web.\\\\
Dazu kommt, dass das Eurotrip plattformunabhängig ist und der Nutzer auch keinen eingerichteten Account auf der Website erstellen muss. Lediglich ein aktueller Browser wird benötigt. Die gesamten Projektdaten sind dabei frei und offen im Internet zur Verfügung gestellt.
\subsection{Semantic Gaming}
Dieser Abschnitt gibt einen kurzen Überblick über den Begriff des Semantic Gaming und beschreibt, welche Teile davon für uns von Interesse sind. Die genaue technische Umsetzung findet sich in Kapitel \ref{sec:gen-struk-daten}.\\\\
Ein grundlegendes Problem des Semantic Webs ist die, verglichen mit anderen Web 2.0 Anwendungen, relativ geringe Nutzerbeteiligung. Eine höhere Nutzerbeteiligung wäre aber von Vorteil, da das Strukturieren der Daten ein Vorgang ist, der zumeist nicht einfach automatisiert werden kann, sondern durch Menschen geschehen muss. Derjenige, der die Daten strukturiert, zieht aber häufig erstmal keinen direkten Nutzen daraus. Zu hoffen, dass das Strukturieren durch reinen Altruismus oder die Bezahlung in irgendeiner Art und Weise befriedigend erfolgt, erscheint wenig vielversprechend. An dieser Stelle setzt das System der \textit{Games with a Purpose for the Semantic Web}\footnote{http://www.computer.org/portal/web/csdl/doi/10.1109/MIS.2008.45} an. Es setzt darauf dem Nutzer bzw. in diesem Fall dem Spieler für das Strukturieren der Daten einen Gegenwert in Form von Spielspaß und einer Art intellektuellen Wettbewerb zu bieten.\\\\
Diese Strukturierung der Daten passiert dabei bei uns unbewusst und im Hintergrund, um den Nutzer nicht aus der Immersion des Spiels zu reißen. Für ihn soll zu jedem Zeitpunkt die Unterhaltung durch die Software im Vordergrund stehen. 
\newpage
\section{Projektplanung}
\subsection{Arbeitspakete und Zuständigkeiten}
Innerhalb des Teams wurde zunächst das Ziel formuliert sowie analysiert, welche Möglichkeiten LOD Datensätze bieten. Ein regelmäßiges, wöchentliches Treffen des Projektteams diente dabei der Projektplanung, zeitlicher und inhaltlicher Kontrolle sowie der Abstimmung und Zusammenführung individuell erarbeiteter Teilmodule.
\begin{figure}[H]
	\centering
	\includegraphics[width=0.8\columnwidth, angle=0]{projektplanung_organigramm_sw.png}
	\caption{Aufteilung der Arbeitspakete}
	\label{pic:Pakete}
\end{figure}
\enlargethispage{2.0cm}
Innerhalb der Gruppe wurden für jede Aufgabe Milestones mit Fristen definiert und einer Person direkt zugeordnet.\footnote{Ein Auszug der Liste mit Arbeitspaket, Milestones, Frist und Zuständigkeit ist in der Zwischenpräsentation enthalten.}
\subsection{Evaluierung}
Als vorteilhaft hat sich herausgestellt keine absolute Frist für die Projektidee zu setzen. Aufgrund von Schwierigkeiten mit LOD Datensätzen musste die Idee inkrementell angepasst werden. Der modulare Aufbau ermöglichte die parallele Ausarbeitung, die mit dem Versionsverwaltungstool Github  effizient gestaltet werden konnte. Weiterhin stellte das wöchentliche Treffen das Kernelement der Projektdurchführung dar. Für unseren Einsatzzweck hat es sich als effektiverer als eine zu zeitlich determinierte Gesamtplanung innerhalb eines Projektplanungstools erwiesen. Geholfen hat dabei die Aufnahme des Arbeitspakets „Organisation“, welches maßgeblich für die effiziente und erfolgreiche Projektkontrolle verantwortlich war.\\\\
Die Projektaufgaben aufgrund unterschiedlicher Ausgangspositionen ausgeglichen zu verteilen, stellte zunächst eine Schwierigkeit dar. Durch klare Verhaltensregeln, spezifische Einarbeitung sowie die Anpassung der Aufgaben an das vorhandene Know-how ist diese Situation allerdings nachhaltig verbessert worden.
\newpage
\section{Umsetzung, Softwarepakete und Programm}
\subsection{Schematischer Aufbau}
Konzeptionell baut das Programm auf einer zentrierten Struktur auf. Dabei ist die Javascript-Engine das Herz der Anwendung. Sie kann auf verschiedene PHP-Dateien zugreifen, die sie quasi wie Methoden benutzt. Die PHP-Dateien hingegen realisieren die Abfragen der LOD-Datensätze. Somit benutzt die Engine reines JSON als Schnittstelle, wohingegen die PHP-Dateien die verschiedenen Schnittstellentechnologien der Datensätze implementieren und zumeist die gefundenen Daten auch für die Engine aufbereiten. Durch die sehr dynamische, heterogene Entwicklung des Semantic Webs ist ein Wechsel der Datensätze oder des Abfragemodus ein realisitsch auftretendes Problem. Unser Design hat hier den Vorteil, dass bei gleichbleibenden Schnittstellen zwischen Engine und PHP-Dateien diverse Änderungen bei der Abfrage von Datensätzen realisiert werden können, ohne dass die eigentliche Engine angepasst werden muss.\footnote{Eine vollständige Konzeptskizze ist im Anhang unter Abschnitt \pageref{pic:Konzept} zu finden}\\\\
Die implementierten PHP-Skripte sind im Detail:
\begin{itemize}
\label{sec:getCity}
\item \textbf{getCity.php:} Durchsucht Geonames mit gewissen Filterkriterien nach Städten. So wird nach europäischen Städten mit mehr als 400000 Einwohnern gesucht\footnote{Für Deutschland wird dieser Filter auf 250000 abgesenkt}, die nicht in Russland oder der Ukraine liegen. Zusätzlich können Städte mit schlechter Datenlage gezielt ausgeschlossen werden.
\item \textbf{getPictures2.php:} Die Engine übergibt dieser PHP-Datei einen validen Stadtnamen\footnote{siehe check.php }. Zu diesem sucht die getPictures dann in der Freebase nach passenden Sehenswürdigkeiten, checkt in der DBpedia ob zu diesen Flickr Wrappr PhotoCollections existieren und holt sich dann zu je drei Sehenswürdigkeiten mit Collection je zwei Fotografien. Diese werden dann, zusammen mit den DBpedia- und Flickr Wrappr-Verlinkungen an die Engine zurückgegeben.
\item \textbf{getInfo.php:} Fragt den englischen Abstract der übergebenen Stadt ab, kürzt ihn auf knapp 200 Zeichen, wobei eine mittige Textstelle ausgewählt wird und gibt diesen String als Ergebnis für den Kurzhinweis zurück.
\label{sec:check}
\item \textbf{check.php:} Diese Datei prüft ob ein übergebener URI ein existierender Bezeichner für einen DBpedia-Eintrag ist. Dies wird genutzt, um ein Grundmaß an Validität bei der Übergabe von Resourcen via URI zu gewährleisten.
\item \textbf{getFlag.php:} Eine Beispielimplementierung für eine Bo"-nus"-fra"-gen"--Me"-tho"-de. Diese bekommen immer drei Länder übergeben und müssen zu diesen drei Antwortmöglichkeiten ausgeben. Dies vollführt diese Datei mit Hilfe des Flaggen-Thumbnails der Länder bei der DBpedia.
\item \textbf{writeResultXML.php:} Schreibt die während dem Spielablauf gesammelten Daten strukturiert in eine XML-Datei.
\end{itemize}
\subsection{Verwendete Technologien und Frameworks}
Neben dem LOD-Ansatz des Semantic Web benutzt das Programm selbst noch weitere Technologien und Frameworks.\\\\
Auf der Clientseite läuft im Browser einer Webapplikation, deren Benutzeroberfläche mit HTML, CSS und JavaScript realisiert wurde. Die eigentliche Engine ist mit der jQuery Bibliothek\footnote{http://www.jquery.com} in JavaScript geschrieben. Sie steuert das Spielgeschehen und verbindet die Programmteile miteinander. Das Kartenmaterial wird über die Google Maps API\footnote{http://code.google.com/intl/de/apis/maps/} eingebunden. Mit Hilfe dieser API wird auch die Entfernungsberechnung anhand des vom Spieler angeklickten Ortes durchgeführt. Zur Kommunikation zwischen den Programmteilen wird auf das JSON Format gesetzt. Serverseitig fungieren PHP Skripte als Datenprovider. Die Anwendung ist somit auf jedem Webserver mit PHP Unterstützung lauffähig. Während der Entwicklung nutzten wir das XAMPP Paket\footnote{http://www.apachefriends.org/de/xampp.html}, welches einen lokalen Apache Webserver mitliefert.
\subsection{Verwendete LOD-Datensätze}
Die verwendeten LOD-Datensätze sind:
\begin{itemize}
\item \textbf{Geonames} wird eingesetzt, um eine Auswahl an Städten zu erzeugen, die der Spieler später dann lokalisieren muss. Diese Datenquelle wurde gegenüber der DBpedia bevorzugt, da hier die Städte klarer strukturiert abgespeichert sind. So fällt die Abfrage der Städte und deren Länder leichter, da es eine eindeutige Hierarchie von Kontinent, Land und Stadt gibt. 
\item Die \textbf{DBpedia} wird gleich an mehreren Stellen im Programm verwendet. Grundsätzlich gilt, dass wenn Namen von Städten, Ländern oder Sehenswürdigkeiten zwischen verschiedenen Programmteilen ausgetauscht werden, dann müssen diese mit ihrem DBpedia URI übergeben werden, um die Validität zu gewährleisten. Weiterhin werden sowohl der Kurzhinweis als auch die beispielhaft implementierte Bonusfrage mit Hilfe der DBpedia generiert. Auch wird dort mit dem dafür zuständigen Tag geprüft, ob eine Sehenswürdigkeit oder Stadt eine Flickr Wrappr PhotoCollection besitzt.
\item \textbf{Freebase} dient vor Allem der Optimierung der Foto-Suchergebnisse. Da sich rein mit einer Stadt getaggte Bilder als teilweise ungenügend herausgestellt haben, wird Freebase zunächst nach bekannten touristischen Sehenswürdigkeiten der einzelnen Städte abgefragt, da zu diesen i.d.R. aussagekräftigere Fotos zu finden sind.
\item Der \textbf{Flickr Wrappr} basiert auf einem PHP-Skript der FU Berlin. Man übergibt diesem Skript als Übergabeparameter einen Stadt- oder Sehenswürdigkeitsnamen, welcher dem URI der DBpedia entsprechen sollte. Der Flickr Wrappr gibt als Suchergebnis strukturierte Daten in Form einer XML/RDF Datei zurück, aus der die Links zu den gesuchten Bildern extrahiert werden können.
\item Im Zuge der Bonusfrage können außerdem noch \textbf{weitere Datensätze} in das bestehende Programm integriert werden. Solange diese die gleiche Schnittstelle wie die existierende Bonusfrage benutzen, können sie ohne weitere Probleme in das Programm integriert werden.
\end{itemize}
\subsection{Stabilität und Verfügbarkeit}
Das Programm an sich läuft trotz seines Prototypen-Status recht stabil. Allerdings ist die Erreichbarkeit von drei Datenquellen für den grundlegenden Ablauf der Fragerunden zwingend erforderlich. Diese sind Geonames, die DBpedia sowie der Flickrwrappr. Ein Ausfall der Freebase würde lediglich die Qualität der Bilder senken, da dann das Heraussuchen von Touristischen Attraktionen wegfallen würde. Sollte einer der Datenbanken der Bonusfragen nicht antworten, wäre ein Fallback auf eine andere Bonusfrage oder das komplette Deaktivieren denkbar. Bei beiden Möglichkeiten ist weiterhin ein stabiler, wenn auch eingeschränkter Betrieb des Programms möglich.\\\\ 
Nach unseren Beobachtungen war Geonames nahezu immer erreichbar, wäh"-rend bei DBPedia temporäre Ausfälle zu verzeichnen waren. Der Flickr Wrappr funktionierte die meiste Zeit problemlos, wenn auch sehr langsam. Ende Juni allerdings war er schlagartig nicht mehr erreichbar, was laut seinem Entwickler auf wiederholte Botattacken zurückzuführen ist. Er empfahl uns eine lokale Kopie seines Programmes zu verwenden, was wir von da an taten.
\subsection{Generierung strukturierter Datensätze}
\label{sec:gen-struk-daten}
Ein weiterer technischer Aspekt ist die Generierung eigener, strukturierter Daten. Dies ist einer der Grundgedanken sowohl des Semantic Webs als auch des Semantic Gamings. Unser Programm ist dabei in der Lage schon rudimentär neu von uns verknüpfte Daten in Kombination mit Daten, die wir durch die Eingabe der Nutzer gewonnen haben, als strukturierte Daten abzuspeichern. Dabei wird für jede Stadt jeweils ein Eintrag in einem XML-Datensatz angelegt, der befüllt wird mit den aktuellen Sehenswürdigkeiten, Bildlinks sowie der vom Nutzer erreichten Entfernung zum Zielort.\\\\ Dadurch erschaffen wir eine Datenbank aus der man durch geschicktes Abfragen eine ganz neue Qualität an Information beziehen kann. So sind wir in der Lage bei ausreichend großen Nutzerdateninput durch gemittelte Entfernungen Aussagen über den Bekanntheitsgrad von Städten im Allgemeinen, aber auch deren Sehenswürdigkeiten zu treffen. Denkbar wäre z.B. auch das Blacklisting einzelner unrelevanter Bilder, die eine signifikante Abweichung in der erreichten Entfernung zu anderen Bildern der gleichen Sehenswürdigkeit aufweisen.\\\\
Wir sind damit in der Lage das bei LOD grundlegend heikle Thema der Bewertung von Elementen innerhalb einer Cloud zu verbessern, da man nun mit der Entfernung zumindest ein qualitatives Maß für die Bekanntheit dieser Städte und Sehenswürdigkeiten besitzt.\\\\
In Eurotrip generiert so jeder Spieler genau die Daten, die wir zu Beginn der Entwicklung gesucht haben, welche aber in der Form noch nicht vorhanden waren. Dies dürfte dem Grundgedanken des Semantic Webs entsprechen, da dieses seine Mächtigkeit ja gerade durch die nahezu unendliche Möglichkeit der Kombination und Verknüpfung von bestehenden und das Einpflegen von neuen strukturierten Daten bezieht.
\subsection{Installation und Betrieb}
Das Spiel selbst benötigt keine clientseitige Installation. Einzig ein funktionierender, aktueller Browser\footnote{getestet in Internet Explorer 9, Firefox 3.5+, Google Chrome 11+, Safari 5}, der in der Lage ist Javascript auszuführen, ist von Nöten. Der Start erfolgt durch Ansteuerung der Webadresse.\\\\
Für den Betrieb des Spiels ist der Anbieter des Programms allerdings gezwungen einen Webserver zu betreiben. Dieser muss in der Lage sein PHP zu interpretieren und so die nötigen Abfragen auf die verschiedenen Datenquellen auszuführen.
\newpage
\section{Lessons learned}
\subsection{Erkenntnisse}
Das wichtigste Ergebnis unserer Projektarbeit ist, dass ein Spiel mit LOD-Datensätzen machbar ist. Insbesondere ist dies möglich, ohne dass die positiven Eigenschaften eines Spiels vernachlässigt werden. Auch als Prototyp kann die Anwendung dem Benutzer richtig Spaß machen.\\\\
Für die Anwendung als Spiel macht der Einsatz von LOD dort am meisten Sinn, wo strukturierte Datenabfragen gefragt sind. Besonders bei Spielideen mit Wissensabfragen ist das der Fall. Umgekehrt macht die Umsetzung eines Spiels für das Semantic Web Sinn, wenn die Benutzer mit Spaß am Spiel dazu beitragen, strukturierte Daten zu generieren. Aus dem Nutzerinput und den gesammelten Datenverlinkungen kann dann ein Mehrwert entstehen. Im Falle von Eurotrip „erzeugt“ das Spielen quasi eine neuartige Cloud mit allen im Spiel gesammelten Informationen. Sogar Aussagen zu deren Relevanz können getroffen werden. Dies ist von Interesse, da man bei der Nutzung von LOD des Öfteren auf das Problem der Relevanz und Priorisierung stößt.\\\\
Grundsätzlich entsteht dann ein Mehrwert, wenn verschiedene Datenbanken effizient miteinander verknüpft und  gekoppelt werden. Die Effizienz der LOD Nutzung lässt sich erhöhen, indem man vom separaten „realtime lookup“ zum konsequenten  „Abwandern“ von Verlinkungen übergeht.  So sind wir von der getrennten Abfrage von Sehenswürdigkeiten und Bildern auf eine verbundene Abfrage übergegangen, in der der Link von Freebase zu DBpedia und von dort weiter zur Flickr Wrappr PhotoCollection führt. Leider sind diese Wege nicht immer konsistent verfügbar. Auch für dieses Problem kann der Output unserer Anwendung Abhilfe schaffen.\\\\
Die technischen Voraussetzungen sind relativ leicht erlernbar, wenn Vorwissen in Programmiersprachen (wie z.B. Java, C++ etc.) verfügbar ist. Angenehm ist auch, dass der Technologieeinsatz sehr flexibel ist und so gut an das Know-how des Teams angepasst werden kann. Die Verwendung von PHP für die Programmierung der Skripte hat sich als gute Wahl herausgestellt, da die Einarbeitung im verfügbaren Zeitrahmen möglich war. Bei der Zusammenführung und im Umgang mit Fehlern hat es sich als hilfreich erwiesen, Projektmitglieder mit ausreichendem Vorabwissen im Team zu haben. Die Kombination von Javascript, AJAX und PHP erwies sich als kompatibel und funktionsfähig. Erschwerend war der Umgang mit den uneinheitlichen Standards bezüglich der Abfragemethode (Sparql, JSON, XML/RDF). \\\\
Bei der technischen Umsetzung im Team war die frühe Einigung auf Schnittstellenformate und Übergabeparameter für die Übersichtlichkeit maßgeblich verantwortlich. Ebenso das Versionsverwaltungstool Github. Im Verlauf der Zusammenarbeit identifizierten wir folgende 3 drei Faktoren als ausschlaggebend für den Erfolg: 
\begin{itemize}
\item uneingeschränktes Commitment und Verantwortungsbewusstsein 
\item aktives Nachfragen im Fall von Unklarheiten sowie 
\item kontinuierliche Kontrolle des Projektfortschrittes.
\end{itemize} 
Wie erwartet und in der Planung berücksichtigt wurde für Feinabstimmung und Tests am Abschluss viel Zeit verwendet. 
\subsection{Vor- und Nachteile von Linked Open Data}
In diesem Kapitel werden die Erfahrungen unseres Projektteams mit dem Umgang mit LOD-Datensätzen evaluiert. Die Aussagen beziehen sich dabei gezielt auf jene Datensätze, die wir für unser Spiel benutzt haben. \\\\
\textbf{Vorteile:}
\begin{itemize}
\item LOD-Datensätze sind Open Source und somit kostenlos für jedermann verfügbar. Eine schnell wachsende Datenmenge bietet Informationen in diversen Quellen zu einer Vielzahl verschiedener Themen. Kein abgeschlossenes System wäre in der Lage in dieser Geschwindigkeit zum Wissenszuwachs beizutragen (n-n Beziehung). Durch die dezentrale Speicherung ist das Datenvolumen nahezu unbegrenzt. Bei der Benutzung von LOD sind geringe Eintrittsbarrieren zu überwinden, da die „Ansteuerung“  leicht erlernbar sowie der Technologieeinsatz flexibel gestaltet werden kann. 
\item Im Gegensatz zu konventioneller Informationsrepräsentation ermöglicht die Semantik eine strukturelle Kategorisierung. Als Beispiel erlaubt der Übergang von wikipedia.org auf dbpedia.org eine Abfrageform hinsichtlich der gewählten Hierarchie.  Durch Querverlinkungen zu verwandten, bereits existierenden Daten sind alle LOD miteinander verknüpft. Durch das „Abwandern“ dieser Links entsteht ein optimaler Zugang zu allen verfügbaren Daten, die für verschiedenste Anwendungen genutzt werden können.
\item Die Verantwortung der Datenpflege liegt beim Benutzer. Falls die in diesem Projekt generierten Outputdaten online verfügbar gestellt wür"-den, könnte es das LOD Netz erweitern. Die Kombination von Geo"-names-Städtelinks, DBpedia-Einträgen, Freebase-Informationen zu Sehens"-würdigkeiten und die Ergebnisse des Nutzerverhaltens bezüglich des Erkennungsgrad der gewählten Bilder stellt einen neuartigen Mehrwert dar, der anderen Nutzern zur Verfügung gestellt werden kann.
\end{itemize}
\textbf{Nachteile:}
\begin{itemize}
\item Dezentrale Erstellung und Erweiterung von Datensystemen (bottom-up) führen zu gravierenden Mängeln. Datensätze sind zum Teil falsch oder unvollständig. Die Strukturierung in der DBpedia ist zum Beispiel bezüglich der Stadtklassifikation uneinheitlich\footnote{populated place, city, town, settlement} und die Datenrepräsentation inkonsistent\footnote{Österreichische Städte haben eine Zuordnung zum Land, deutsche Städte hingegen nicht}. Durch fehlende Datenpflege finden sich des Öfteren Verlinkungen ins Nichts. Bei der Verwendung mehrere Datenquellen werden unterschiedliche Formate und Konventionen bezüglich Leer- und Sonderzeichen verwendet, sodass eine Vielzahl von Fehlern abgefangen werden müssen.
\item Größere Schwierigkeiten gibt es bei der Frage nach der Relevanz von Daten. Flickrwrappr liefert immer Daten jedoch ist die Bewertung der Bilder bezüglich der Erkennbarkeit schwierig. Hier könnten die von uns im Spiel ermittelten Daten Abhilfe schaffen. Auch der Bekanntheitsgrad der Musikgruppen, die DBtunes für Deutschland anzeigt, ist z.B. sehr gering. 
\item Die Nutzung von Verlinkungen zwischen LOD-Datensätzen ist schwierig, da diese nur unvollständig verfügbar sind. DBpedia gibt Links zu Factbook an, es können jedoch keine Links aus Geonames direkt extrahiert werden.   
\item Bei der Nutzung von LOD ist man per Definition internetabhängig und von Geschwindigkeit und Ausfallrisiko der Server betroffen. DBpedia und Flickr Wrappr fielen auch im Untersuchungszeitraum teilweise tagelang aus. Der „Wettbewerbsvorteil“ durch LOD ist relativ instabil, da jede Idee leicht kopierbar ist und dies ja auch durchaus gewünscht ist. 
\end{itemize}
\newpage
\section{Fazit}
\subsection{Das Projektergebnis „in a nutshell“}
Die Anwendung zieht aus der Verknüpfung von mehreren LOD Datensätzen einen direkten Nutzen. Zur Entwicklung der Frage werden Daten gebündelt und existierende Querverweise zwischen Datensätzen genutzt. Für das Spiel eignen sich LOD Daten, da sie strukturell abfragbar sind. Durch die positiven Begleiterscheinungen des Semantic Gaming werden gebündelte Datensätze mit im Spiel erzeugten Daten verknüpft. Das Ergebnis kann ein Beitrag zur Lösung des Qualitätsproblems von LOD Datenquellen liefern.
\begin{figure}[H]
	\centering
	\includegraphics[width=0.75\columnwidth, angle=0]{vor_nachteile_sw.png}
	\caption{Vor- und Nachteile des Programms}
	\label{pic:Pakete}
\end{figure}
\enlargethispage{2.0cm}
\subsection{Ausblick}
Die Spielanwendung ist bewusst so aufgebaut, dass sie flexibel erweiterbar ist. Als Erweiterung zum klassischen Fragetyps sowie der Bonusfrage sind viele Konzepte denkbar. DBtunes enthält eine umfangreiche Sammlung von Musikgruppen, deren Herkunft über Koordinaten abzufragen sind. Jedoch ist der Bekanntheitsgrad der Bands sehr gering. Als Bonusfrage können weitere Fragen auf Länderebene, z.B. nach Küstenlänge, Nachbarstaaten, bekannte Persönlichkeiten, entstehen. Wenn man sich von der von uns gewählten Spielsituation löst, sind auch Quizformen in der Form von „Wer wird Millionär“ denkbar. Eine Output-orientierte Erweiterung, die für uns aus thematischen und zeitlichen Gründen nicht möglich war, ist die konsequente Auswertung der erzeugten Daten. XML erlaubt die Möglichkeit von Abfragen, um beispielsweise die Bekanntheit von Städten oder die Güte von Bildern zu ermitteln. Dieses Instrument kann zur Verbesserung der Bilderdatenbank verwendet werden. Auch das Aussortieren qualitativ nichtssagender Bilder, z.B. via Tagging, wäre vorstellbar. Ein Hauptaugenmerk sollte auf Qualitätsentwicklung der Datensätze gelegt werden. Einheitliche Strukturen, Fehlereliminierung und eine bessere Serververfügbarkeit würden die Arbeit mit LOD Datensätzen deutlich erleichtern.
\newpage
\setcounter{page}{2}
\pagenumbering{Roman}
\section{Anhang}
\begin{figure}[H]
  	\hspace{20pt}
	\centering
	\includegraphics[width=1.7\columnwidth, angle=90]{seminarLOD.png}
	\caption{Konzeptskizze}
	\label{pic:Pakete}
\end{figure}
% das ist wohl jetzt das Ende des Dokumentes
\end{document}